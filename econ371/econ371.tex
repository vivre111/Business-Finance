\documentclass[10pt]{article} 

\usepackage{fullpage}
\usepackage{bookmark}
\usepackage{amsmath}
\usepackage{amssymb}
\usepackage[dvipsnames]{xcolor}
\usepackage{hyperref} % for the URL
\usepackage[shortlabels]{enumitem}
\usepackage{mathtools}
\usepackage[most]{tcolorbox}
\usepackage[amsmath,standard,thmmarks]{ntheorem} 
\usepackage{physics}
\usepackage{pst-tree} % for the trees
\usepackage{verbatim} % for comments, for version control
\usepackage{tabu}
\usepackage{tikz}
\usepackage{float}

\lstnewenvironment{python}{
\lstset{frame=tb,
language=Python,
aboveskip=3mm,
belowskip=3mm,
showstringspaces=false,
columns=flexible,
basicstyle={\small\ttfamily},
numbers=none,
numberstyle=\tiny\color{Green},
keywordstyle=\color{Violet},
commentstyle=\color{Gray},
stringstyle=\color{Brown},
breaklines=true,
breakatwhitespace=true,
tabsize=2}
}
{}

\lstnewenvironment{cpp}{
\lstset{
backgroundcolor=\color{white!90!NavyBlue},   % choose the background color; you must add \usepackage{color} or \usepackage{xcolor}; should come as last argument
basicstyle={\scriptsize\ttfamily},        % the size of the fonts that are used for the code
breakatwhitespace=false,         % sets if automatic breaks should only happen at whitespace
breaklines=true,                 % sets automatic line breaking
captionpos=b,                    % sets the caption-position to bottom
commentstyle=\color{Gray},    % comment style
deletekeywords={...},            % if you want to delete keywords from the given language
escapeinside={\%*}{*)},          % if you want to add LaTeX within your code
extendedchars=true,              % lets you use non-ASCII characters; for 8-bits encodings only, does not work with UTF-8
% firstnumber=1000,                % start line enumeration with line 1000
frame=single,	                   % adds a frame around the code
keepspaces=true,                 % keeps spaces in text, useful for keeping indentation of code (possibly needs columns=flexible)
keywordstyle=\color{Cyan},       % keyword style
language=c++,                 % the language of the code
morekeywords={*,...},            % if you want to add more keywords to the set
% numbers=left,                    % where to put the line-numbers; possible values are (none, left, right)
% numbersep=5pt,                   % how far the line-numbers are from the code
% numberstyle=\tiny\color{Green}, % the style that is used for the line-numbers
rulecolor=\color{black},         % if not set, the frame-color may be changed on line-breaks within not-black text (e.g. comments (green here))
showspaces=false,                % show spaces everywhere adding particular underscores; it overrides 'showstringspaces'
showstringspaces=false,          % underline spaces within strings only
showtabs=false,                  % show tabs within strings adding particular underscores
stepnumber=2,                    % the step between two line-numbers. If it's 1, each line will be numbered
stringstyle=\color{GoldenRod},     % string literal style
tabsize=2,	                   % sets default tabsize to 2 spaces
title=\lstname}                   % show the filename of files included with \lstinputlisting; also try caption instead of title
}
{}

% floor, ceiling, set
\DeclarePairedDelimiter{\ceil}{\lceil}{\rceil}
\DeclarePairedDelimiter{\floor}{\lfloor}{\rfloor}
\DeclarePairedDelimiter{\set}{\lbrace}{\rbrace}
\DeclarePairedDelimiter{\iprod}{\langle}{\rangle}

\DeclareMathOperator{\Int}{int}
\DeclareMathOperator{\mean}{mean}

% commonly used sets
\newcommand{\R}{\mathbb{R}}
\newcommand{\N}{\mathbb{N}}
\newcommand{\Q}{\mathbb{Q}}
\renewcommand{\P}{\mathbb{P}}

\newcommand{\sset}{\subseteq}

\theoremstyle{break}
\theorembodyfont{\upshape}

\newtheorem{thm}{Theorem}[subsection]
\tcolorboxenvironment{thm}{
enhanced jigsaw,
colframe=Dandelion,
colback=White!90!Dandelion,
drop fuzzy shadow east,
rightrule=2mm,
sharp corners,
before skip=10pt,after skip=10pt
}

\newtheorem{cor}{Corollary}[thm]
\tcolorboxenvironment{cor}{
boxrule=0pt,
boxsep=0pt,
colback={White!90!RoyalPurple},
enhanced jigsaw,
borderline west={2pt}{0pt}{RoyalPurple},
sharp corners,
before skip=10pt,
after skip=10pt,
breakable
}

\newtheorem{lem}[thm]{Lemma}
\tcolorboxenvironment{lem}{
enhanced jigsaw,
colframe=Red,
colback={White!95!Red},
rightrule=2mm,
sharp corners,
before skip=10pt,after skip=10pt
}

\newtheorem{ex}[thm]{Example}
\tcolorboxenvironment{ex}{% from ntheorem
blanker,left=5mm,
sharp corners,
before skip=10pt,after skip=10pt,
borderline west={2pt}{0pt}{Gray}
}

\newtheorem*{pf}{Proof}
\tcolorboxenvironment{pf}{% from ntheorem
breakable,blanker,left=5mm,
sharp corners,
before skip=10pt,after skip=10pt,
borderline west={2pt}{0pt}{NavyBlue!80!white}
}

\newtheorem{defn}{Definition}[subsection]
\tcolorboxenvironment{defn}{
enhanced jigsaw,
colframe=Cerulean,
colback=White!90!Cerulean,
drop fuzzy shadow east,
rightrule=2mm,
sharp corners,
before skip=10pt,after skip=10pt
}

\newtheorem{prop}[thm]{Proposition}
\tcolorboxenvironment{prop}{
boxrule=0pt,
boxsep=0pt,
colback={White!90!Green},
enhanced jigsaw,
borderline west={2pt}{0pt}{Green},
sharp corners,
before skip=10pt,
after skip=10pt,
breakable
}

\setlength\parindent{0pt}
\setlength{\parskip}{2pt}


\begin{document}
\let\ref\Cref

\title{\bf{econ371 Business Finance 1}}
\date{\today}
\author{Austin Xia}

\maketitle
\newpage
\tableofcontents
\listoffigures
\listoftables
\newpage
\section{Basic Concepts in Finance}
\subsection{The Three Types of Firms}
\begin{itemize}
    \item Sole Proprietorships: Owned and run by one person, limitation: no seperation between firm and owner
    \item Partnerships: \begin{itemize}
        \item general partnership: unincorporated business owned/run by more than 1 owner, owners have same right
        \item limited partnership: owner's liability is limited to their investment, A limited partner has no management authority
        \item limited liability partnership: for law and accounting firms, provides partial limitation of a partner's liability
    \end{itemize}
    \item Corporation \\ legally defined, seperated from its owners.
    \\ Solely responsible for its own obligations. Owners no liable for any obligations
    \\ ownership stake of corporation is divided into shares \textbf{stock}. The collection of all outstanding shares of a corporation is known as \textbf{equity} of the corporation
    \\ owner of share of stock is \textbf{shareholder/stockholder/equity holder}, shareholders are entitled to dividend payment 
    \\ no limitation on who can be shareholder
    \\ Corporate profit are subject to taxation seperate from its owners' tax obligations, so shareholders pay personal income tax after profit tax
\end{itemize}

Canada Revenue Agency allowed an exemption for double taxation \textbf{flow-through entities (income trust)}
\begin{itemize}
    \item Business income trusts 
    \item energy trusts 
    \item real estate investment trust (REIT)
\end{itemize}
    REIT continue to have no tax while others are now taxed

\begin{defn}[flow-through entity]
    A business in which all income produced flows to the investors and virtually no earnings are retained within the business 
\end{defn}

\begin{defn}[income trust]
    A trust that holds income-producing assets directly or holds all the debt and equity securities of an income-producing corporation within the trust
\end{defn}

\begin{defn}[business income trust]
    A income trust holds all the debt and equity securities of a corporation (the underlying business)
\end{defn}

\begin{defn}[energy trust]
    an income trust that holds resource porperties directly or holds all the debt and equity securities of a recource corporation within the trust
\end{defn}

\begin{defn}[unit holder]
    the owner of an income trust
\end{defn}

\begin{defn}[REIT real estate investment trust]
    an income trust that holds real estate properties directly or holds all the debt and securities of a corporation that owns real estate properties
\end{defn}

\subsection{the financial manager}
\begin{itemize}
    \item make investment decision. main type is production. expanding. doing stuff instead of buying profolio  
    \item make financing decision. how to get money. cooperate loan, sell equity, new share, this will drop stock price
    \item manage short-term cash needs. inflation
\end{itemize}
\subsection{goal of financial manager}
\begin{itemize}
    \item the overriding goal is maximize the wealth of stockholder/increase long-term value to company
\end{itemize}
\subsection{financial manager's place in cooperation}
the corporate management team:
\begin{itemize}
    \item stockholders elect a board of directors 
    \item the board make rules 
    \item the ceo implement rules and policies set by the board
\end{itemize}

Ethics and incentives in corporation: 
\textbf{principle-agent problem}
    managers can put their self-interest before shareholders

to solve this, tie the top manager's conpensation to stock price/corporation profit

\begin{defn}[board of director]
    a group of people elected by shareholders who have ultimate decision making authority in corporation 
\end{defn}

\begin{defn}[ceo cheif executive officer]
    the person charged with running corportaion by instituting roles and policies set by board of directors
\end{defn}

\subsection{stock market}
\textbf{corporation can be private or publick}
\begin{itemize}
    \item private corporation has limited number of owners and no organized market for its shares 
    \item public corporation has many owners and its shares trade on organized market. calling a stock market 
\end{itemize}
private stock cannot be traded, it can be sold back to cooperation but not to public

\textbf{private market vs secondary market}
\begin{itemize}
    \item the market where new shares of stock are issued by corporation and sold to investors 
    \item markets, where shares of corporation are traded between investors, without involvement of the corporation like NYSE
\end{itemize}

\textbf{Bid-Ask Market}
\begin{defn}[bid price]
    the highest price in the market for which one is willing to buy the stock (should be slightly less than market price)
\end{defn}
opposite is lowest ask vs highest bid

\begin{defn}[bid-ask spread]
    an implit \textbf{transaction cost} investors have to pay in order to trade \textbf{quickly}.
    extreme case is buy some stock then sell at once
\end{defn}
stuff: exchange traded funds: similar to mutual fund. weighted average of bucket of stock/currency. 

stuff: liquidity: high liquidity === how easy things can be coverted to money

\begin{defn}[Limit order]
    only until price reach some point, truade it 
\end{defn}

\begin{defn}[market order]
    buy immediately. customer ends up buying at highest ask and selling at lowest bid 
\end{defn}

\subsection{Financial institutions}
\begin{defn}[financial institutions]
    entities that provide financial services. like taking deposit, managing inverstment, brokering financial transactions, making loans.
\end{defn}

the financial cycle:
\begin{itemize}
    \item people invest/save money 
    \item through loans and stock, money goes to companies who use it to fund growth through new products. generating profit and wages 
    \item the money flows back to savers/investors
\end{itemize}

roles of financial institutions:
\begin{itemize}
    \item move funds form savers to borrowers 
    \item move funds through time 
    \item help spread out risk-bearing
\end{itemize}

\section{financial statement}
\subsection{firm's disclosure of financial statement}
\textbf{financial statements} are accounting reports issued periodically to present past performance info and snapshot of firm's asset 

investors, financial analysts managers etc rely on it to obtain info about a corporation

rules of financial statments:\\
generally accepted accounting principle (GAAP) are based on historical cost 
IFRS places more emphasis on fair value of assets/liability

\subsection{the statement of financial position or balance sheet}
List firm's asset, liabilities and equity.

provides a \textbf{snapshot} of the firm's financial position at a give point in time

balance sheet identity: 
$$sharholder's equity = assets - liabilities $$

\begin{defn}[current asset]
    current assets is things that will be converted into cash in one year.

    like Cash, accounts receivable, inventories(how much is used to create inventory)
\end{defn}

long term asset has: \textbf{net property, plant, equipment}

total asset = current assets + long-term assets

Liabilities:
\begin{itemize}
    \item current liabilities \begin{itemize}
        \item accounts payable
        \item notes payable/short-term debuts
    \end{itemize}
    \item Long-term Liabilities \begin{itemize}
        \item long-term debut 
    \end{itemize}
\end{itemize}

shareholder's equity = common stock (money spent when buying stock) + retained earnings 

liquidation value or total shareholder's equity (book number of equity) = total assets - total liabilites 

book value of equity is completely different from market value of equity

book vlaue of equity: 
\begin{itemize}
    \item not good esitmate of true value as an ongoing firm (but sometimes used as estimate of the liquidation value = value left after its assets were sold and Liabilities paid)
    \item an inaccurate assessment of actual value of firm equity 
    \item  often idffers substantially from amount investors are willing to pay for the equity
\end{itemize}

\begin{defn}[market capitalization]
    market capitalization = market value of equity = market price per share * number of share. 

    does not depend on book value of equity
\end{defn}

\textbf{liquidation value}: value of firm after assets sold liability paid 

\textbf{market-to-book ratio}: market value of equity / book value of equity

\textbf{value stocks}: firms with low market-to-book ratios

\textbf{growth stocks}: firms with high market-to-book ratios

we want market value of equity to be larger than book value of equity

\textbf{enterprise value} = market value of equity + debt - cash

because debut is used to expand, cash is just cash 

present value of future cash flows === enterprise value

\textbf{Net Working Capital} = current assets - current Liabilities

\textbf{shareholder's equity/book value of equity} = assets - Liabilities

\textbf{market capitalization} = price per share * shares outstanding

\subsection{income statement}
\textbf{gross profit}= revenues - cost of sales

\textbf{operating income} = gross profit - operating income 

\textbf{earnings before interest and tax EBIT} = operating income + other income 

\textbf{pretax income} = EBIT + interest income

\textbf{net income} = pretax income - tax

\textbf{earning per share} = net income / shares outstanding 

\textbf{diluted EPS} is EPS considering increase of shares by stock option/convertible bonds 

\subsection{statement of cash flow}
contains operating activities, investment activities, financing activities

\textbf{operating activity}

minus increase in accounts receivable,

adds increase in account payable 

minus increase to inventories

add depreciation 

\textbf{investment activity}

subtract captial expenditure (increase in long-term-assets minus depreciation)

subtract investment or purchase

\textbf{financing activity}

minus dividends paid 

adds cash received from sale/repurchasing stock 

changes to short-term/long-term borrowing 

\subsection{income statement analysis}
\subsubsection{profitability ratio}

\textbf{gross margin} = gross profit/sales 

\textbf{operating margin} = operating income / sales 

\textbf{net profit margin} = net income / sales 

\subsubsection{liquidity ratio}
\textbf{current ratio} = current asset / current Liabilities 

\textbf{quick ratio} = (current asset - inventory) / current liability

\subsubsection{asset efficienty}
\textbf{asset turnover} = sales / total asset 

\textbf{fixed asset turnover} = sales / fixed assets

\subsubsection{working capital ratio}
\textbf{accounts receivable days} = accounts receivable / average daily sales 

\textbf{inventory days / inventory turnover} = cost of good / inventory

\textbf{interest coverage ratio} = earnings / interest, how easily firm cover its interest payment 

\subsubsection{leverage ratios}
\textbf{debt-equity ratio} = total debt / total equity 

\textbf{debt-to-capital ratio}

\textbf{net debt} = total debt - excess cash - short-term investment 

\textbf{debut-to-enterprise-value ratio} = net debt / (market value of equity + net debt)

\textbf{enterprise value} = market value of equity + net debt

\textbf{equity multiplier} = total assets / book balue of equity

\subsubsection{valuation ratio}
\textbf{price-earning ratio (P/E)} = market capitalization / net income = share price / earnings per share

\textbf{PEG ratio} = P/E / expected growth rate

\subsubsection{operating returns}
\textbf{return on equity} = net income / book value of equity 

\subsection{investment return}
\textbf{return on asset} = net income / total assets

\textbf{return on invested capical} = EBIT(1-tax rate) / (book value of equity + net debt)

\textbf{DuPont Identity}: ROE = (net income / sales) (sales / total assets) (total assets/total equity)

\section{time value of money}
\subsection{value cash flow at different points of time}
rule 1: comparing and combining values 

rule 2: compounding: calculate future value

rule 3: discounting: calculate value of future cash flow at an earlier point of time


\subsection{perpetuities and annuities}
perpetuities:

C = r * P, P is money in bank, r is interest rate 

PV(C in perpertuity) = $\frac{C}{r}$

annuities: 

PV(N-year annuitiy of C per year with interest r) = $C*\frac{1}{r}\left(1-\frac{1}{(1+r)^N}\right)$

future value of annuitiy is just $PV * (1+r)^n$

if we add inflation it would be PV(growing perpetuity) = $\frac{C}{r-g}$

present value of growing annuity:

$$PV=C_1 * \frac{1}{r-g}\left(1-(\frac{1+g}{1+r})^N\right)$$

\section{interest rate}
\subsection{interest rate quotes and adjustment}
\begin{defn}[EAR effective annual rate]
    the total amount of interest that will be earned at end of one year 
\end{defn}

\begin{defn}[APR annual percentage rates, simple interest]
    indicates amount of interest earned in one year without compounding

    simple interest is interest earned without considering compounding
\end{defn}

$$1+EAR=(1+\frac{APR}{m})^m$$

\subsection{determinants of interest rates}
\begin{defn}
    nominal interest rates: quoted by banks that indicate rate money will grow

    real interest rate: the rate of growth of purchasing power
\end{defn}

$$1+realRate = \frac{1+norminalRate}{1+inflationRate} = growth of money / growth of price$$

\begin{defn}[term structure]
    relationship between investment term and interest rate
\end{defn}

\begin{defn}[yield curve]
    plot of bond yields as function of matuirity date
\end{defn}

\textbf{interest rate determination}: overnight rate: rate at which banks can borrow cash reserves on overnight basis from Bank of Canada

if interest rate expect to rise, long-term interest rate will be higher to attract investors

\subsection{bond terminology}
\begin{defn}[bond indenture]
    a statement of terms of a bond, as well as amounts and dates of all payments
\end{defn}

\begin{defn}[maturity date]
    the final repayment date of a bond
\end{defn}

\begin{defn}[term]
    the time remaining until final repayment date
\end{defn}

\begin{defn}[face value]
    norminal amount used to compute interest payment

    typically repaid at maturity 

    usually standard increments like 1000
\end{defn}

\begin{defn}[coupons]
    the promised interest payments paid periodically until maturity date 
\end{defn}

coupon rate is expressed as APR, 

CPN is each coupon payment
$$CPN=\frac{couponRate * faceValue}{numberOfPayment}$$

\begin{defn}[zero coupon bonds]
   only 2 cash flows 
\end{defn}

yeild to maturity: 
$$1+YTM_n = \frac{faceValue}{price}^{1/n}$$

\section{stocks}
\begin{defn}[common stock]
    share of ownership, gives rights to any common dividends, rights to vote on election
\end{defn}
\begin{defn}[ticker symbol]
    a unique abbreviation assigned to each publicly traded compamny
\end{defn}
\begin{defn}[preferred stock]
    are issued with stated dividend rates, in bankruptcy or dividend, preferred stock ranks higher

    cumulative perferred stock entitle investors to reap any missed dividends
\end{defn}

\begin{defn}[proxy]
    a written authorization for someone else to vote your shares.

    a proxy contest is a contest between 2 or more groups competing to collect proxies to prevail in the matter up for shareholder vote
\end{defn}
\begin{defn}[cumulative vs non-cumulative perfered stock]
    cumulative is all missed preferred dividends must be paid before common dividends may be paid
    
    non-cumulative is current preferred dividends must be paid before common dividends may be paid
\end{defn}
\begin{defn}[dividend yield]
    expected annual divident of stock divided by its current price.
\end{defn}
\begin{defn}[capital gain]
    the amount by which the selling price exceeds initial purchase price

    captical gain rate: $\frac{P1-P0}{P0}$
\end{defn}

\begin{defn}[total return]
    $r_E = {Div_1}{P0}+\frac{P_1-P_0}{P_0}$
    this is dividend yeild - capital gain rate
\end{defn}
    $P_0=\frac{Div_1}{1+r_E}+\frac{Div_2+P_2}{(1+r_E)^2}$

\textbf{constant dividend growth model} some dividends grows at constant rate 

$P_0=\frac{Div_1}{r_E - g}$

\subsection{estimating dividends}
dividends versus investment and growth 

increase dividend in 3 ways 
\begin{itemize}
    \item  increase in earnings 
    \item increase in dividend payout rate 
    \item decrease in number of shares
\end{itemize}

dividend payout rate (fraction of firm's earnings taht firm pays out as dividend)

$Div=\frac{earnings}{sharesOutstanding}* dividendPayoutRate$

change in earnings = new inverstment * return on new investment

new investment = earnings * rentention rate 

if dividend payout rate is constant,

g = retention rate * return on new investment 

\subsection{estimating dividends in dividend discount model}
$$P_n = \frac{Div_{n+1}}{r_E-g}$$

if a firm do share repurchase, Div = Div + share repurchase 

\textbf{discounted free cash flow model}
enterprise value = market value + debt - cash


free cash flow = EBIT * (1-taxRate)+ depreciation - capital expenditure - increase in net working capital

discounted free cash flow model: V = PV(future free cash flow of firm)

$$P_0=\frac{V_0+cash_0-debt_0}{sharesOutstanding}$$

since we are discounting cash flows to all investors, we use \textbf{weighted average cost of capital (WACC)}, $r_{wacc}$

$$V_0=\frac{FCF_1}{1+r_{wacc}}+\frac{FCF_2}{(1+r_{wacc}^2)}...+\frac{FCF_n}{(1+r_{wacc}^n)}+\frac{V_n}{(1+r_{wacc})^n}$$

for estimating the terminal value, we need to ssume a constant long-run growth rate $g_{FCF}$ for free cash flows beyond year N

$$V_N = \frac{FCF_{N+1}}{r_{wacc}-g_{FCF}}= (\frac{1+g_{FCF}}{r_{wacc}-g_{FCF} })FCF_N$$

\subsection{valuation based on comparable firms}
valuation multiples: 
\begin{itemize}
    \item price earnings ratio: most common, share price / earnings per share
    \item enterprise value multiples 
    \item other multiples \begin{itemize}
        \item multiples of sales 
        \item price-to-book value of equity 
        \item industry-specific ratios
    \end{itemize}
\end{itemize}

we calculate P/E ratio using trailing earnings or forward earnings. 

trailing is earnings over prior 12 months 

forward is earnigns over coming 12 months, preferred as we most concerned about future earnings 

forward P/E = $\frac{P_0}{EPS_1}=\frac{DividendPayoutRate(DIV/EPS)}{r_E-g}$

P/E ratio relates only to equity, ignoring debt, enterprise value multiples use a measure of earnings before interest payments are made.
\begin{itemize}
    \item ebit 
    \item ebitda 
    \item free cash flow (because capital expanditure can vary, most common is to use enterprise value to EBITDA multiple)
\end{itemize}

when expected free cash flow growth constant, 
$$V_0/EBITDA_1=\frac{\frac{FCF_1}{r_wacc-g_{FCF}}}{EBITDA_1}=\frac{FCF_1/EBITDA_1}{r_{wacc}-g_{FCF}}$$

\section{Fundamentals of Capital Budgeting}
\begin{defn}[capital budget]
    a list of projects that a company plans to undertake during next period
\end{defn}

\begin{defn}[capital budgeting]
    process of analyzing investing opportunities and deciding which ones to accept
\end{defn}

\begin{defn}[incremental earnings]
    the amount by which a firm's earnings are expected to change as a result of an investment decision
\end{defn}

incremental revenue and cost estimates: facts to consider:
\begin{itemize}
    \item a new product has lower sales initailly 
    \item average price and cost of production changes overtime 
    \item competition tens to reduce profit
\end{itemize}

operating expenses vs captical expenditure:

cost of plant property and equipment is divided to be deducted when estimating earnings

capital cost allowance (for tax purpose)

the incremental cca deduction claimed at end of tax year is undepreciated capital cost multipled by CCA rate

$CCA_t = UCC_t*d$

$UCC_1 = 0.5* CapEx$
$UCC_t = CApEx * (1-d/2)*(1-d)^{t-2}$

incremental revenue and cost estimates 

incremental EBIT = incremental revenue / incremental cost - CCA

CCA: capital cost allowance: canada revenue agency method of depreciation for income tax purpose 

CCA rate: the proportion of underpreciated capital cost that can be claimed as cca in a given tax year

half-year rule: as assets may be purchased any time throughout a year, it can be assumed on average, an asset is owned for half a year during the first tax year of its ownership

undepreciated capital cost (UCC): the balance, at a point of time, calculated by deducting an asset's current and prior CCA amounts from original cost of the asset

incremental earnings forecast: 
\begin{itemize}
    \item pro forma statement: a statement that is not based on actual data but rather depicts a firm's financials under a given set of hypothetical assumptions
    \item taxes and negative ebit 
    \item interest expense: unlevered net income: net income that does not include 
\end{itemize}

converting from earnings to free cash flow(the incremental effect of a project on a firm's available cash)

capital expenditures and depreciation

net working capital = current assets - current liabilities = cash + inventory + receivables - payables

\textbf{trade credit} the difference between receivables and payables is net amount of firm's capital that is consumed as a result of these credit transactions
\end{document}
